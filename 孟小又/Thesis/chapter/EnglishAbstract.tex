\englishabstract{
In order to improve the effect of scenes in games and virtual reality applications, developers often add natural phenomena such as fog and volumetric light to the scene. The main content of this paper is to use the physics-based way to achieve real-time rendering of fog and volumetric light. Physics-based algorithms for rendering fog and volumetric light usually have problems in efficiency. In addition, related algorithms cannot be compatible with multi-light source scenes, transparent scenes, forward shading rendering pipelines, or deferred shading pipelines. our algorithm mainly solves the above problem of physics-based simulation of fog and volumetric light.

The algorithms of simulated fog and volumetric light with high attention in recent years are based on the Ray-marching algorithm and shadow mapping algorithm. This article takes this method as the research object, and analyzes the theoretical basis of the algorithm based on physical fog and volumetric light in detail. As well as the advantages and disadvantages of effect and efficiency, on the basis of full study of the Ray-Marching algorithm and shadow mapping algorithm, we propose a more efficient Ray-Marching algorithm and a better rendering shadow mapping algorithm. The core idea of this algorithm is to decouple the illumination calculation part of the scattering phenomenon from the Ray-marching step, and use the compute shader to accelerate the lighting calculation process. The results of fog and volumetric light are stored in three-dimensional texture which is a intermediate memory. The results are the input of the rendering process which facilitate the rendering effect of complex scenes, and we uses the exponential shadow mapping technique to optimize the effect of the cascaded shadowing mapping algorithm.

The main works of this paper are summarized as follows:

\begin{enumerate}
  \item we use compute shader to increase the efficiency of the light calculation part of the scattering phenomenon and improve the efficiency of the Ray-marching algorithm;

  \item Our algorithm Supports the rendering of volumetric fog and volumetric light for multi-light source scenes, eliminating the need for preprocessing the light source and improving rendering efficiency;
  \item We use exponential shadow map technology to improve the quality of cascaded shadow maps, solve edge aliasing which is the problem of low resolution, and solve the problem of shading effect which caused by the shift of the shadow, This problem results in the instability of volumetric light, and improve the effect of real-time volumetric light rendering;
  \item Our algorithm has good compatibility and is compatible with the forward shading pipeline, the deferred shading pipeline, and the scene with transparent objects.
\end{enumerate}

The experimental results show that the algorithm for simulating fog and volumetric light in this paper can achieve better rendering results in different scenes, and the algorithm does not require extra calculation or rendering steps for multi-light source scenes and transparent object scenes. This algorithm uses the compute shader to speed up the Ray-marching algorithm. It also enables real-time rendering in scenes with multi-light source and numerous models. Exponential shadow mapping technology can solve the problem of shadow edge aliasing and shadow jitter when the camera is changed in the scene. On this basis, the rendering effect of volumetric light is better, and this algorithm can meet the needs of practical applications.

\keywords{Ray-marching; Scattering; Real-time rendering; 3D textures; shadow maps}

}
